\documentclass{article}
\usepackage[utf8]{inputenc}
\usepackage{float}
\usepackage{makecell}

\title{MI-FME Cvičení 9}
\author{Tomáš Chvosta}
\date{Březen 2020}

\setcounter{secnumdepth}{-2} % no numbered sections
\usepackage{czech}
\begin{document}

\maketitle

Převeďte každou z následujících základních cest na formu SSA (Static Single Assignment Form) a zapište si její
podmínku ověření. Zkontrolujte, zda platí podmínka ověření. Pokud platí, napište krátký důkaz. Pokud tomu tak není, najděte protipříklad, to znamená proměnné přiřazení, které ukazuje, že vzorec neplatí, a které zároveň představuje počáteční stav, který ale po následném provedení vede k chybě.

\section{Cvičení 9a}

\subsection{Zadání:}
    \textbf{assume} $x \geq 0 \newline y \leftarrow x \newline z \leftarrow y \newline @ \ z \geq 0$


\subsection{Static Single Assignment forma:}

V tomto případě není potřeba zavádět nové názvy proměnných, jelikož žádná z proměnných nenabývá podruhé nové hodnoty. SSA forma má tedy následující tvar:\newline\newline \textbf{assume} $x \geq 0 \newline y \leftarrow x \newline z \leftarrow y \newline @ \ z \geq 0$


\subsection{Logická formule z SSA:}

Logická formule z SSA vypadá následovně:
$$(\forall x, y, z )([ x \geq 0 \wedge y = x \wedge z = y ] \Rightarrow z \geq 0) $$

\subsection{Ověřovací podmínky:}
Pro důkaz použijeme metodu ručního dokazování z přednášky. Předpokládáme $x \geq 0$, $y = x$, $z = y$ a dokážeme $z \geq 0$. Kvůli předpokladu $z = y$ stačí dokázat $y \geq 0$. Kvůli předpokladu $y = x$ stačí dokázat $x \geq 0$, což je ale zároveň jeden z předpokladů, takže formule platí a můžeme tvrdit, že program je napsán korektně.

\section{Cvičení 9b}

\subsection{Zadání:}
    \textbf{assume} $x \geq 0 \newline z \leftarrow y \newline y \leftarrow x \newline @ \ z \geq 0$
    

\subsection{Static Single Assignment forma:}

Můžeme si všimnout, že ve druhém kroku přiřadíme hodnotu $y$ do $z$ a následně přiřadíme do $y$ novou hodnotu $x$. Je tedy potřeba  zavést nový název proměnné pro toto přiřazení. SSA forma má tedy následující tvar:\newline\newline \textbf{assume} $x \geq 0 \newline z \leftarrow y \newline y_1 \leftarrow x \newline @ \ z \geq 0$


\subsection{Logická formule z SSA:}
Logická formule z SSA vypadá následovně:
$$(\forall x, y, y_1, z )([ x \geq 0 \wedge z = y \wedge y_1 = x ] \Rightarrow z \geq 0) $$


\subsection{Ověřovací podmínky:}
Jelikož je na první pohled zřejmé, že formule nebude logicky platná, pokusíme se najít protipříklad a tím dokázat, že formule neplatí. Můžeme využít předpokladu $z = y$ a dosadit ho do pravé strany formule. Získáme tedy závěr, že $y \geq 0$. Tedy pro $y < 0$ závěr očividně neplatí. Protipříklad tedy může vypadat například takto:
$$\{x \mapsto 8, y \mapsto -5, y_1 \mapsto 8, z \mapsto -5 \}$$
Toto ohodnocení proměnných splňuje levou stranu formule, avšak nesplňuje pravou stranu. Z toho můžeme usoudit, že formule neplatí. Toto ohodnocení klidně může být i počátečním stavem. Můžeme si pro představu ukázat běh programu:

\begin{table}[H]\centering

\begin{tabular}{|c|c|c|c|}
    
        \hline \textbf{Krok} & $x$ & $y$ & $z$ \\ \hline \hline
    	\textbf{assume} $x \geq 0$ & 8 & -5  & \\  \hline
    	$z \leftarrow y$ & 8 & -5 & -5 \\ \hline
    	$y \leftarrow x$ & 8 & 8 & -5 \\ \hline
    	$@ \ z \geq 0$ & 8 & 8 & -5 \\ \hline
    	
    	
    	\end{tabular}
\end{table}

Z tabulky si můžeme všimnout, že ve chvíli, kdy má být hodnota proměnné $z \geq 0$ je $z = -5$. Program tedy není korektní.

\section{Cvičení 9c}

\subsection{Zadání:}

\textbf{assume} $x < 0 \newline x \leftarrow x - k \newline$ \textbf{assume}  $ k \leq 1 \newline @ \ x \geq 0$
    

\subsection{Static Single Assignment forma:}

Můžeme si všimnout, že na začátku předpokládáme nějakou hodnotu $x < 0$ a poté ve druhém kroku přiřadíme hodnotu $x - k$ do $x$. Je tedy potřeba  zavést nový název proměnné pro toto přiřazení. SSA forma má tedy následující tvar:\newline\newline \textbf{assume} $x < 0 \newline x_1 \leftarrow x - k \newline$ \textbf{assume}  $ k \leq 1 \newline @ \ x_1 \geq 0$

\subsection{Logická formule z SSA:}
Logická formule z SSA vypadá následovně:
$$ (\forall x, x_1, k)([x < 0 \wedge x_1 = x - k \wedge k \leq 1] \Rightarrow x_1 \geq 0) $$

\subsection{Ověřovací podmínky:}

Při dokazování, že platí $x_1 \geq 0$ můžeme využít předpokladu $x_1 = x - k$ a dokazovat $ x - k \geq 0$. To můžeme pomocí základních matematických pravidel upravit na tvar $x \geq k$. To znamená, že stačí najít protipříklad, který bude splňovat $ x < k$. Protipříklad tedy může vypadat například takto:
$$\{x \mapsto -5, x_1 \mapsto -1, k \mapsto -4 \}$$
Toto ohodnocení proměnných splňuje levou stranu formule, avšak nesplňuje pravou stranu. Z toho můžeme usoudit, že formule neplatí. Můžeme si pro představu ukázat běh programu:

\begin{table}[H]\centering

\begin{tabular}{|c|c|c|}
    
        \hline \textbf{Krok} & $x$ & $k$ \\ \hline \hline
    	\textbf{assume} $x < 0$ & -5 & -4 \\  \hline
    	 $x \leftarrow x - k$ & -1 & -4 \\ \hline
    	 \textbf{assume} $k \leq 1$ & -1 & -4 \\ \hline
    	 $@ \ x \geq 0$ & -1 & -4 \\ \hline
    	
    	\end{tabular}
\end{table}

Z tabulky si můžeme všimnout, že ve chvíli, kdy má být hodnota proměnné $x \geq 0$ je $x = -1$. Program tedy není korektní.

\section{Cvičení 9d}

\subsection{Zadání:}

\textbf{assume} $k \leq x \newline x \leftarrow x - k \newline @ \ x \geq 0$

\subsection{Static Single Assignment forma:}
Můžeme si všimnout, že ve druhém kroku přiřazujeme proměnné $x$ novou hodnotu. Je tedy potřeba zavést nový název proměnné pro toto přiřazení. SSA forma má tedy následující tvar:\newline\newline \textbf{assume} $k \leq x \newline x_1 \leftarrow x - k \newline @ \ x_1 \geq 0$

\subsection{Logická formule z SSA:}
Logická formule z SSA vypadá následovně:
$$ (\forall x, x_1, k)([k \leq x \wedge x_1 = x - k] \Rightarrow x_1 \geq 0) $$

\subsection{Ověřovací podmínky:}
Pro důkaz použijeme metodu ručního dokazování z přednášky. Předpokládáme $k \leq x$, $x_1 = x - k$ a dokážeme $x_1 \geq 0$. Kvůli předpokladu $x_1 = x - k$ stačí dokázat $x - k \geq 0$, tedy $x \geq k$. To však triviálně dokazuje předpoklad $k \leq x$. Můžeme tedy tvrdit, že program je napsán korektně.

\section{Cvičení 9e}

\subsection{Zadání:}

$x \leftarrow x - k \newline$ \textbf{assume} $k \leq x \newline @ \ x \geq 0$

\subsection{Static Single Assignment forma:}

Můžeme si všimnout, že v prvním kroku přiřazujeme proměnné $x$ novou hodnotu. Je tedy potřeba zavést nový název proměnné pro toto přiřazení. SSA forma má tedy následující tvar:\newline\newline $x_1 \leftarrow x - k$ \newline \textbf{assume} $k \leq x_1 \newline @ \ x_1 \geq 0$

\subsection{Logická formule z SSA:}

Logická formule z SSA vypadá následovně:
$$ (\forall x, x_1, k)([x_1 = x - k \wedge k \leq x_1] \Rightarrow x_1 \geq 0) $$

\subsection{Ověřovací podmínky:}
Při dokazování, že platí $x_1 \geq 0$ můžeme využít předpokladu $x_1 = x - k$ a dokazovat $ x - k \geq 0$. To můžeme pomocí základních matematických pravidel upravit na tvar $x \geq k$. Zároveň můžeme získat nový předpoklad složení předpokladů $x_1 = x - k$, $k \leq x_1$. Získáme předpoklad $k \leq x - k$ tedy $2k \leq x$. Máme tedy předpoklad $2k \leq x $ a závěr $ x \geq k$. Můžeme si však všimnout, že nejspíše bude existovat nějaké $x$ a nějaké $k$, pro které bude splněn předpoklad, ale nebude platit závěr. Pojďmě ho najít. Hledáme tedy protipříklad, kdy $2k \leq x \wedge x < k$ tedy $2k \leq x < k$. Pro úpravě získáme, že $k < 0$. Zvolme tedy například $k = -1$ a $x$ tak, aby platilo $-2 \leq x < -1$, tedy například $x = -2$. Protipříklad tedy může vypadat například takto:
$$\{x \mapsto -2, x_1 \mapsto -1, k \mapsto -1 \}$$
Toto ohodnocení proměnných splňuje levou stranu formule, avšak nesplňuje pravou stranu. Z toho můžeme usoudit, že formule neplatí. Můžeme si pro představu ukázat běh programu:

\begin{table}[H]\centering

\begin{tabular}{|c|c|c|}
    
        \hline \textbf{Krok} & $x$ & $k$ \\ \hline \hline
        initial state & -2 & -1 \\ \hline
        $x \leftarrow x - k$ & -1 & -1 \\ \hline
        \textbf{assume} $k \leq x$ & -1 & -1 \\ \hline
        $@ \ x \geq 0$ & -1 & -1 \\ \hline
    	
    	\end{tabular}
\end{table}

Z tabulky si můžeme všimnout, že ve chvíli, kdy má být hodnota proměnné $x \geq 0$ je $x = -1$. Program tedy není korektní.

\section{Cvičení 9f}

\subsection{Zadání:}
\textbf{assume} $ k \geq 0 \newline x \leftarrow x - k \newline$ \textbf{assume} $k \leq x \newline @ \ x \geq 0$

\subsection{Static Single Assignment forma:}
Můžeme si všimnout, že ve druhém kroku přiřazujeme proměnné $x$ novou hodnotu. Je tedy potřeba zavést nový název proměnné pro toto přiřazení. SSA forma má tedy následující tvar:\newline\newline \textbf{assume} $k \geq 0 \newline x_1 \leftarrow x - k \newline $ \textbf{assume} $ k \leq x_1 \newline @ \ x_1 \geq 0$

\subsection{Logická formule z SSA:}
Logická formule z SSA vypadá následovně:
$$ (\forall x, x_1, k)([k \geq 0 \wedge x_1 = x - k \wedge k \leq x_1] \Rightarrow x_1 \geq 0) $$

\subsection{Ověřovací podmínky:}
Předpokládáme $k \geq 0$, $x_1 = x - k$, $k \leq x_1$ a dokážeme $x_1 \geq 0$. Spojením prvního a třetího předpokladu získáme předpoklad $0 \leq k \leq x_1$. Závěr $x_1 \geq 0$ je ekvivalentní s $\neg \neg (x_1 \geq 0)$. Jelikož dokazujeme negaci, můžeme předpokládat $\neg (x_1 \geq 0)$ tedy $x_1 < 0$ a najít spor. Spojením předpokladu $0 \leq k \leq x_1$ s předpokladem $x_1 < 0$ získáme předpoklad $0 \leq k \leq x_1 < 0$ tedy po úpravě $0 < 0$, což je spor. Formule tedy platí a můžeme tvrdit, že program je napsán korektně.

\section{Cvičení 9g}

\subsection{Zadání:}
\textbf{assume} $x \geq 0 \newline y \leftarrow x \newline$ \textbf{input} $x \newline @ \ x \geq 0$

\subsection{Static Single Assignment forma:}
Jelikož ve třetím kroku načítáme do $x$ novou hodnotu, je potřeba zavést nový název proměnné. SSA forma má tedy následující tvar:\newline\newline \textbf{assume} $x \geq 0 \newline y \leftarrow x \newline $ \textbf{input} $ x_1  \newline @ \ x_1 \geq 0$

\subsection{Logická formule z SSA:}
Logická formule z SSA vypadá následovně:
$$ (\forall x, x_1, y)([x \geq 0 \wedge y = x \wedge \top] \Rightarrow x_1 \geq 0) $$

\subsection{Ověřovací podmínky:}
Na pravé straně formule máme závěr $x_1 \geq 0$, nicméně v předpokladech není $x_1$ nijak omezeno. Je to způsobeno tím, že bezprostředně po načtení nové hodnoty do $x_1$ uživatelským vstupem, má být $x_1 \geq 0$. Můžeme tedy velmi snadno najít protipříklad, pro který formule neplatí, například:$$\{x \mapsto 8, y \mapsto 8, x_1 \mapsto -1 \}$$
Toto ohodnocení proměnných splňuje levou stranu formule, avšak nesplňuje pravou stranu. Z toho můžeme usoudit, že formule neplatí. Můžeme si pro představu ukázat běh programu:

\begin{table}[H]\centering

\begin{tabular}{|c|c|c|}
    
        \hline \textbf{Krok} & $x$ & $y$ \\ \hline \hline
        \textbf{assume} $x \geq 0$ & 8 & \\ \hline
        $y \leftarrow x$ & 8 & 8 \\ \hline
        \textbf{input} $x$ & -1 & 8  \\ \hline
        $@ \ x \geq 0$ & -1  & 8  \\ \hline
    	
    	\end{tabular}
\end{table}

Z tabulky si můžeme všimnout, že ve chvíli, kdy má být hodnota proměnné $x \geq 0$ je $x = -1$. Program tedy není korektní.

\section{Cvičení 9h}

\subsection{Zadání:}
$y \leftarrow x \newline$ \textbf{input} $x \newline$ \textbf{assume} $ y \geq 0 \newline @ \ y \geq 0$

\subsection{Static Single Assignment forma:}
Jelikož ve druhém kroku načítáme do $x$ novou hodnotu, je potřeba zavést nový název proměnné. SSA forma má tedy následující tvar:\newline\newline $y \leftarrow x \newline $ \textbf{input} $ x_1  \newline$ \textbf{assume} $y \geq 0 \newline @ \ y \geq 0$

\subsection{Logická formule z SSA:}
Logická formule z SSA vypadá následovně:
$$ (\forall x, x_1, y)([y = x \wedge \top \wedge y \geq 0] \Rightarrow y \geq 0) $$

\subsection{Ověřovací podmínky:}
Máme předpoklady $y = x$, $y \geq 0$ a máme dokázat $y \geq 0$. Vidíme, že $y \geq 0$ je zároveň předpoklad i závěr, formule je tedy triviálně dokázána a můžeme tvrdit, že program je napsán korektně.

\section{Cvičení 9i}

\subsection{Zadání:}
$y \leftarrow x \newline$ \textbf{input} $x \newline$ \textbf{assume} $ x \geq 0 \newline @ \ y \geq 0$

\subsection{Static Single Assignment forma:}
Jelikož ve druhém kroku načítáme do $x$ novou hodnotu, je potřeba zavést nový název proměnné. SSA forma má tedy následující tvar:\newline\newline $y \leftarrow x \newline $ \textbf{input} $ x_1  \newline$ \textbf{assume} $x_1 \geq 0 \newline @ \ y \geq 0$

\subsection{Logická formule z SSA:}
Logická formule z SSA vypadá následovně:
$$ (\forall x, x_1, y)([y = x \wedge \top \wedge x_1 \geq 0] \Rightarrow y \geq 0) $$

\subsection{Ověřovací podmínky:}
Máme předpoklady $y = x$, $x_1 \geq 0$ a máme dokázat $y \geq 0$. Kvůli předpokladu $y = x$ dokazujeme $x \geq 0$. Dále však nemáme žádný předpoklad, který by nějak omezoval $x$. Stačí tedy najít protipříklad takový, že $x < 0$. Protipříklad tedy může vypadat například takto:$$\{x \mapsto -1, y \mapsto -1, x_1 \mapsto 8 \}$$ Toto ohodnocení proměnných splňuje levou stranu formule, avšak nesplňuje pravou stranu. Z toho můžeme usoudit, že formule neplatí. Můžeme si pro představu ukázat běh programu:

\begin{table}[H]\centering

\begin{tabular}{|c|c|c|}
    
        \hline \textbf{Krok} & $x$ & $y$ \\ \hline \hline
        $y \leftarrow x$ & -1 & -1  \\ \hline
        \textbf{input} $x$ & 8 & -1  \\ \hline
        \textbf{assume} $x \geq 0$ & 8 & -1 \\ \hline
        $@ \ y \geq 0$ & 8  & -1 \\ \hline
    	
    	\end{tabular}
\end{table}

Z tabulky si můžeme všimnout, že ve chvíli, kdy má být hodnota proměnné $y \geq 0$ je $y = -1$. Program tedy není korektní.

\end{document}
