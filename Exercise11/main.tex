\documentclass{article}
\usepackage[utf8]{inputenc}
\usepackage{float}
\usepackage{makecell}
\newcommand\tab[1][0.5cm]{\hspace*{#1}}
\newcommand{\N}{\mathbb{N}}
\usepackage{mathtools}
\usepackage{amssymb}

\usepackage{amsfonts}

\title{MI-FME Cvičení 11}
\author{Tomáš Chvosta}
\date{Duben 2020}

\setcounter{secnumdepth}{-2} % no numbered sections
\usepackage{czech}
\begin{document}

\maketitle

Uvažujte následující specifikaci programu:

\begin{itemize}
    \item Input: $x, y \in \N_0, y \geq 1$
    \item Output: $xy$
\end{itemize}
$r \leftarrow x \newline i \leftarrow 1 \newline$ \textbf{while} $i < y$ \textbf{do} $\newline \tab @ \newline \tab r \leftarrow r + x \newline \tab i \leftarrow i + 1 \newline @\ r = xy \newline$ \textbf{return} $r$ 

\section{Cvičení 11a}

\subsection{Zadání}
Vytvořte tabulku, která bude ukazovat hodnoty proměnných $r$ a $i$ v místě programu s chybějící assertací ve všech iteracích cyklu pro nějaké netriviální vstupy $x$ a $y$.

\subsection{Řešení}
Nejprve pojďme zvolit dvě libovolné hodnoty $x$ a $y$ (samozřejmě tak, aby platilo, že $x, y \in \N_0, y \geq 1$). Například $x = 10$ a $y = 5$. Následující tabulka zobrazuje hodnoty proměnných $i$ a $r$ v místě programu s chybějící assertací pro celý běh programu:
\begin{table}[H]\centering
    \begin{tabular}{||c|c||}
            \hline $i$ & $r$  \\ \hline \hline
    	    1 & 10 \\  \hline
    	    2 & 20 \\  \hline
    	    3 & 30 \\  \hline
    	    4 & 40 \\  \hline
    \end{tabular}
\end{table}

Nyní si pojďme ukázat, jak bude tabulka vypadat pro obecné hodnoty $x, y$ a $n \in \{1, ..., y-1\}$ :
\begin{table}[H]\centering
    \begin{tabular}{||c|c||}
            \hline $i$ & $r$  \\ \hline \hline
    	    1 & $x$ \\  \hline
    	    2 & $2x$ \\  \hline
    	    3 & $3x$ \\  \hline
    	    \makecell{.\\ .\\. } & \makecell{.\\ .\\. } \\  \hline
    	    $n$ & $nx$ \\  \hline
    	    \makecell{.\\ .\\. } & \makecell{.\\ .\\. } \\  \hline
    	    $y - 1$ & $( y - 1 )x$ \\  \hline
    \end{tabular}
\end{table}


\section{Cvičení 11b}

\subsection{Zadání}
Pokuste se odhadnout chybějící assertaci uvnitř cyklu.

\subsection{Řešení}
Z předchozích tabulek se můžeme pokusit odhadnout, který výraz bychom mohli doplnit do assertace na začátku cyklu. Z tabulky lze vypozorovat následující vlastnosti:
\begin{itemize}
    \item $r = ix$
    \item $i < y$
\end{itemize}
Pojďme tedy tyto dvě vlastnosti doplnit do assertace na začátku cyklu. Doplněním vznikne následující program:$\newline\newline r \leftarrow x \newline i \leftarrow 1 \newline$ \textbf{while} $i < y$ \textbf{do} $\newline \tab @\ r = ix \wedge i < y \newline \tab r \leftarrow r + x \newline \tab i \leftarrow i + 1 \newline @\ r = xy \newline$ \textbf{return} $r$

\section{Cvičení 11c}

\subsection{Zadání}
Zapište odpovídající základní cesty programu a zkontrolujte, zda platí jejich ověřovací podmínky.

\subsection{Řešení}
Z programu v předchozí sekci lze vytvořit následující základní cesty programu.

\subsubsection{Základní cesta 1}
$r \leftarrow x \newline i \leftarrow 1 \newline$ \textbf{assume} $\neg(i < y) \newline @\ r = xy$ 

\subsubsection{SSA forma 1}
V tomto případě není třeba zavádět nové názvy proměnných.

\subsubsection{Logická formule 1}
$$(\forall x,y,r,i \in \N_0, y \geq 1)([r = x \wedge i = 1 \wedge \neg(i<y)] \Rightarrow r = xy )$$

\subsubsection{Ověřovací podmínka 1}
Máme předpoklady $r = x$, $i = 1$, $\neg(i < y)$ a máme dokázat $r = xy$. Z předpokladů $\neg(i < y)$ a $i = 1$ můžeme vytvořit předpoklad $y \leq 1$. Jelikož však zadání definuje, že $y \geq 1$, může $y$ nabývat pouze hodnoty $y = 1$. Použijeme tedy předpoklady $r = x$ a $ y = 1$ a dosadíme je do dokazované části formule $r = xy$, čímž dostaneme $x = x$, což je triviálně dokázáno. Ověřovací podmínka pro základní cestu 1 tedy platí.

\subsubsection{Základní cesta 2}
$r \leftarrow x \newline i \leftarrow 1 \newline$ \textbf{assume} $i < y \newline @\ r = ix \wedge i < y$ 

\subsubsection{SSA forma 2}
V tomto případě není třeba zavádět nové názvy proměnných.

\subsubsection{Logická formule 2}
$$(\forall x,y,r,i \in \N_0, y \geq 1)([r = x \wedge i = 1 \wedge i < y] \Rightarrow [r = ix \wedge i < y]  )$$

\subsubsection{Ověřovací podmínka 2}
Máme předpoklady $r = x$, $i = 1$, $i < y$ a máme dokázat $r = ix \wedge i < y$, tedy dokázat zvlášť $r = ix$ a $i < y$. Můžeme si všimnout, že $i < y$ je zároveň i předpoklad, tedy tuto část máme triviálně dokázanou. Nyní použijeme předpoklady $r = x$ a $ i = 1$ a dosadíme je do dokazované části formule $r = ix$, čímž dostaneme $x = x$, což je triviálně dokázáno. Ověřovací podmínka pro základní cestu 2 tedy platí.

\subsubsection{Základní cesta 3}
\textbf{assume} $r = ix \wedge i < y \newline \tab r \leftarrow r + x \newline \tab i \leftarrow i + 1 \newline$ \textbf{assume} $i < y \newline @\ r = ix \wedge i < y$  

\subsubsection{SSA forma 3}
\textbf{assume} $r = ix \wedge i < y \newline \tab r_1 \leftarrow r + x \newline \tab i_1 \leftarrow i + 1 \newline$ \textbf{assume} $i_1 < y \newline @\ r_1 = i_1x \wedge i_1 < y$  

\subsubsection{Logická formule 3}
$$(\forall x,y,r,i \in \N_0, y \geq 1)$$ $$([r = ix \wedge i < y \wedge r_1 = r + x \wedge i_1 = i + 1 \wedge i_1 < y] \Rightarrow [r_1 = i_1x \wedge i_1 < y]  )$$

\subsubsection{Ověřovací podmínka 3}
Máme předpoklady $r = ix$, $i < y$, $r_1 = r + x$, $i_1 = i + 1$, $i_1 < y$ a máme dokázat $r_1 = i_1x \wedge i_1 < y$, tedy dokázat zvlášť $r_1 = i_1x$ a $i_1 < y$. Můžeme si všimnout, že $i_1 < y$ je zároveň i předpoklad, tedy tuto část máme triviálně dokázanou. Z~předpokladů $r = ix$ a $r_1 = r + x$ můžeme vytvořit nový předpoklad $r_1 = (i + 1)x$. Do tohoto předpokladu můžeme dosadit předpoklad $i_1 = i + 1$, čímž vznikne předpoklad $r_1 = i_1x$, což jsme zároveň měli dokázat. Ověřovací podmínka pro základní cestu 3 tedy platí.

\subsubsection{Základní cesta 4}
\textbf{assume} $r = ix \wedge i < y \newline \tab r \leftarrow r + x \newline \tab i \leftarrow i + 1 \newline$ \textbf{assume} $\neg(i < y) \newline @\ r = xy$

\subsubsection{SSA forma 4}
\textbf{assume} $r = ix \wedge i < y \newline \tab r_1 \leftarrow r + x \newline \tab i_1 \leftarrow i + 1 \newline$ \textbf{assume} $\neg(i_1 < y) \newline @\ r_1 = xy$
\subsubsection{Logická formule 4}
$$(\forall x,y,r,i \in \N_0, y \geq 1)$$ $$([r = ix \wedge i < y \wedge r_1 = r + x \wedge i_1 = i + 1 \wedge \neg(i_1 < y)] \Rightarrow r_1 = xy )$$

\subsubsection{Ověřovací podmínka 4}
Máme předpoklady $r = ix$, $i < y$, $r_1 = r + x$, $i_1 = i + 1$, $\neg(i_1 < y)$ a máme dokázat $r_1 = xy$. Z předpokladů $r = ix$ a $r_1 = r + x$ můžeme vytvořit nový předpoklad $r_1 = (i + 1)x$. Do tohoto předpokladu můžeme dosadit předpoklad $i_1 = i + 1$, čímž vznikne nový předpoklad $r_1 = i_1x$. Dále můžeme získat z~ předpokladu $\neg(i_1 < y)$ předpoklad $y \leq i_1$, který můžeme následně spojit s předpokladem $i < y$, čímž získáme předpoklad $i < y \leq i_1$ z čehož zjistíme, že $y = i_1$. Tento nový předpoklad můžeme dosadit do předpokladu $r_1 = i_1x$ a tím zíksat $r_1 = xy$, což jsme zároveň měli dokázat. Ověřovací podmínka pro základní cestu 4 tedy platí.

\section{Cvičení 11d}

\subsection{Zadání}
Pokud jsou všechny základní cesty programu korektní, stejně tak jako celý algoritmus, pak je úkol dokončen. V opačném případě upravte assertaci uvnitř cyklu a pokračujte předchozími dvěma úkoly, dokud nejsou všechny základní cesty korektní.

\subsection{Řešení}
V předchozí sekci si můžeme všimnout, že všechny základní cesty programu jsou korektní a tím tedy i celý algoritmus.

\end{document}
