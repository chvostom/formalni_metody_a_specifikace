\documentclass{article}
\usepackage[utf8]{inputenc}
\usepackage{mathtools}
\usepackage{amssymb}

\usepackage{amsfonts}
\newcommand{\Z}{\mathbb{Z}}
\newcommand{\N}{\mathbb{N}}

\title{MI-FME Cvičení 2}
\author{Tomáš Chvosta}
\date{Únor 2020}

\setcounter{secnumdepth}{-2} % no numbered sections

\begin{document}

\maketitle

\section{Cvičení 2a}

\subsection{Zadání:}

Navrhněte input/output specifikaci pro problém nalezení prvočíselného rozkladu pro zadané celé číslo.

\subsection{Vstup:}
Celé číslo $ n \in \Z , n \geq 2 $

\subsection{Výstup:}
n-tice prvočísel $ (p_1, p_2, \dots, p_k)$ taková, že platí:
$$(\forall i \in \Z)(1 \leq i \leq k)(prvocislo(p_i) \wedge \prod_{i=1}^k p_i = n) $$
$$(\forall p \in \Z)(prvocislo(p)):\Leftrightarrow  (p > 1) \wedge ((\forall x \in \N)((p\ mod\ x = 0) \Rightarrow (x = 1) \lor (x = p))) $$

\subsection{Poznámky:}
Chci prvočíselný rozklad, takže prostě hledám čísla, která budou prvočísla a~zároveň, když je všechna vynásobím, tak získám to číslo, které bylo zadáno. Potom musím zadefinovat prvočíslo, což musí být celé číslo větší než 1 a pokud existuje číslo, které ho dělí beze zbytku, pak je to buď jednička nebo to samotné číslo.

\section{Cvičení 2b}

\subsection{Zadání:}

Navrhněte input/output specifikaci pro vyhledávání řetězce.

\subsection{Vstup:}
Pole znaků $text$ délky $n$ \newline Pole znaků $hledany$ délky $k$

\subsection{Výstup:}
Pravda - v případě, že se hledaný řetězec vyskytuje v zadaném textu. \newline
Nepravda - v případě, že se hledaný řetězec nevyskytuje v zadaném textu.

$$(\exists i \in \Z)(0 \leq i \leq n - k)(\forall j \in \Z)(0 \leq j < k)(text[i+j] = hledany[j])$$

\subsection{Poznámky:}

V této úloze vlastně hledám index v textu, kde začíná hledaný řetězec. Mělo by se mi to do toho textu vejít, proto je tam $i \leq n - k$. Když ten index mám, tak už jen stačí zkontrolovat jednotlivé znaky, jestli jsou stejné.


\section{Cvičení 2c}

Navrhněte input/output specifikaci pro kontrolu, zda se jeden řetězec shoduje s~druhým, který obsahuje právě jeden žolíkový znak '*'.

\subsection{Zadání:}

\subsection{Vstup:}
Pole znaků $reg$ délky $n$ \newline
Pole znaků $text$ délky $m$

\subsection{Výstup:}

Pravda - v případě, že se řetězce shodují. \newline
Nepravda - v případě, že se řetězce neshodují.

$$(\exists i \in \Z)(reg[i] =\ '*') \wedge (((\forall j \in \Z)(0 \leq j < i)(reg[j] = text[j])) \ \wedge $$
$$((\forall k \in \Z)(i < k < n)(reg[k] = text[m-n+k]))) \ \wedge \ (m \geq n - 1)$$

\subsection{Poznámky:}

V řetězci $reg$ musí být někde právě jedna hvězdička. Tu stačí najít a potom zkontrolovat znaky, které jí předchází a které za ní následují, jestli se vyskytují na správných pozicích v řetězci $text$. Zároveň musí platit $m \geq n - 1$, například když má řetězec $reg$ délku 5, znamená to, že má 4 znaky, které nepředstavují hvězdičku a tyto znaky se nutně musí vyskytovat v řetězci $text$. Z toho můžeme vidět, že délka řetězce $text$ musí být alespoň 4, jinak řetezce $req$ a $text$ nemohou být stejné.

\section{Cvičení 2d}

\subsection{Zadání:}

Navrhněte input/output specifikaci pro algoritmus, který ze zadaného textu vrátí ta slova, která představují palindromy.

\subsection{Vstup:}

Pole znaků $text$ délky $n$

\subsection{Výstup:}

Množina slov z textu: 
$$\{slovo\ |\ palindrom\_v\_textu(text,n,slovo,i,l)\}$$ \newline
$$(\forall text,n,i,l)(palindrom\_v\_text(text,n,slovo,i,l)):\Leftrightarrow $$
$$ oddeleni\_slova(text,n,i,l)\ \wedge \ slovo=substring(text,i,l)\ \wedge \ palindrom(slovo,l)$$ \newline

$$(\forall text,n,i,l)(oddeleni\_slova(text,n,i,l)): \Leftrightarrow $$
$$ (0 \leq i < n - l) \ \wedge $$
$$ ((i=0) \ \lor \  (text[i-1]=\ '\ '))\ \wedge $$
$$ ((i+l=n-1) \ \lor \ (text[i+l+1]=\ '\ ')) \ \wedge $$
$$ (\nexists x \in \Z)(i \leq x \leq i+l)(text[x]=\ '\ ')$$ \newline

$$(\forall text,i,l)(substring(text,i,l):=s)\ \text{s.t.}\ (\forall x \in \Z)(0 \leq x \leq l-1)(s[x]=text[i+x])$$\newline

$$(\forall slovo,l)(palindrom(slovo,l)): \Leftrightarrow (\forall i \in \Z)(0 \leq i < l)(slovo[i]=slovo[l-i-1])$$

\subsection{Poznámky:}
Ve své podstatě jen hledám slova, která splňují vlastnost palindromu. Nejprve je tedy potřeba zkontrolovat, že před slovem a za slovem je mezera (s~výjimkou slova na pozici 0 a slova na úplném konci textu). To přesně dělá $oddeleni\_slova$. Dále funkce $substring$ vrátí jednotlivá slova. Nakonec stačí pouze zkontrolovat, že platí vlastnost palindromu, tedy že na $i$-té pozici je stejný znak jako na $(l-i-1)$-té pozici.\newline
\begin{flushleft}
$text$ \dots zadaný text \newline
$n$ \dots délka zadaného textu \newline
$slovo$ \dots slovo v textu \newline
$i$ \dots pozice slova v textu \newline
$l$ \dots délka slova \newline\newline
\end{flushleft}

\end{document}
