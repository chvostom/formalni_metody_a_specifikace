\documentclass{article}
\usepackage[utf8]{inputenc}
\usepackage{float}
\usepackage{makecell}

\title{MI-FME Cvičení 3}
\author{Tomáš Chvosta}
\date{Únor 2020}

\setcounter{secnumdepth}{-2} % no numbered sections
\begin{document}

\maketitle

\section{Cvičení 3a}

\subsection{Zadání:}

Dokažte následující formuli:
$$[\neg[[r \lor s] \Rightarrow q ] \wedge [[r \lor s] \Rightarrow q]] \Rightarrow [[p \Rightarrow q] \wedge \neg [p \Rightarrow q]]$$

\subsection{Důkaz:}

Jelikož se jedná o implikaci, předpokládáme, že platí levá strana pravidla tedy konjunkce $\neg[[r \lor s] \Rightarrow q ]$ a $[[r \lor s] \Rightarrow q]]$. Tyto dva předpoklady představují $\bot$ a jelikož z $\bot$ plyne cokoliv, nezáleží na tom, co máme na pravé straně implikace a formule vždy platí.

\begin{table}[H]\centering

    \caption{Důkazová tabulka}

\begin{tabular}{|c|c|c|}
    
    % \\ je novej radek | je cara \hline je horizontální čara
    
        \hline \textbf{Krok} & \textbf{Předpokládáme} & \textbf{Dokazujeme} \\ \hline \hline
    	1. & \makecell{$\neg[[r \lor s] \Rightarrow q ]$ \\ $[[r \lor s] \Rightarrow q]]$ \dots $\bot$} & \makecell{$\ \ [p \Rightarrow q]$ \\ $\neg [p \Rightarrow q]$} \\ \hline
    	
    
            
    	\end{tabular}
\end{table}

\section{Cvičení 3b}

\subsection{Zadání:}
Dokažte následující formuli:
$$[\neg p \Rightarrow p] \Rightarrow p$$

\subsection{Důkaz:}
Jelikož se jedná o implikaci, je předpoklad $[\neg p \Rightarrow p]$ a pokusíme se dokázat $p$. Použijeme Ratschanovo důkazní pravidlo, které říká, že když chceme dokázat $p$, pak můžeme nahradit $p$ za $\neg \neg p$ a následně použít pravidlo pro dokazování negací. Do seznamu předpokladů tedy přidáme $\neg p$ a pokusíme se najít spor. Předpoklad $[\neg p \Rightarrow p]$ říká, že musí platit $p$ jelikož máme v předpokladech $\neg p$, což je spor. Spor podle Ratschana dokončí úspěšně jakýkoliv důkaz.

\begin{table}[H]\centering

    \caption{Důkazová tabulka}

\begin{tabular}{|c|c|c|}
    
    % \\ je novej radek | je cara \hline je horizontální čara
    
        \hline \textbf{Krok} & \textbf{Předpokládáme} & \textbf{Dokazujeme} \\ \hline \hline
    	1. & $[\neg p \Rightarrow p]$ & $p$ \\ \hline
    	2. & $\neg p$ & $\neg \neg p$ tedy $p$ \\ \hline
    	3. & $p$ \dots $\bot$ & \\ \hline
    
            
    	\end{tabular}
\end{table}

\section{Cvičení 3c}

\subsection{Zadání:}

Dokažte následující formuli:
$$\neg [p \Rightarrow q] \Rightarrow [q \Rightarrow p]$$

\subsection{Důkaz:}

Jelikož se jedná o implikaci, je $\neg [p \Rightarrow q]$ předpoklad. Pokusíme se tedy dokázat $[q \Rightarrow p]$. Použijeme stejný postup a předpokládáme, že platí $q$. Nyní by se mohlo hodit dokázat, že platí $[p \Rightarrow q]$. Jako lemma tedy zvolíme $[p \Rightarrow q]$ a díky předpokladu $q$ je jasné, že toto lemma platí. Můžeme tedy přidat předpoklad $[p \Rightarrow q]$, což společně s předpokladem $\neg [p \Rightarrow q]$ vytvoří $\bot$, ze které plyne cokoliv.

\begin{table}[H]\centering

    \caption{Důkazová tabulka}

\begin{tabular}{|c|c|c|}
    
    % \\ je novej radek | je cara \hline je horizontální čara
    
        \hline \textbf{Krok} & \textbf{Předpokládáme} & \textbf{Dokazujeme} \\ \hline \hline
    	1. & $\neg [p \Rightarrow q]$ & $[q \Rightarrow p]$ \\ \hline
    	2. & $q$ & $p$ \\ \hline
    	3. &  & lemma $[p \Rightarrow q]$ \\ \hline
    	4. & $[p \Rightarrow q]$ \dots $\bot$  &  \\ \hline
    
            
    	\end{tabular}
\end{table}

\section{Cvičení 3d}

\subsection{Zadání:}

Dokažte následující formuli:
$$$$

\subsection{Důkaz:}

\section{Cvičení 3e}

\subsection{Zadání:}

Dokažte následující formuli:
$$$$

\subsection{Důkaz:}

\section{Cvičení 3f}

\subsection{Zadání:}

Dokažte následující formuli:
$$$$

\subsection{Důkaz:}

\section{Cvičení 3g}

\subsection{Zadání:}

Dokažte následující formuli:
$$$$

\subsection{Důkaz:}

\section{Cvičení 3h}

\subsection{Zadání:}

Dokažte následující formuli:
$$$$

\subsection{Důkaz:}

\section{Cvičení 3i}

\subsection{Zadání:}

Dokažte následující formuli:
$$$$

\subsection{Důkaz:}

\end{document}
