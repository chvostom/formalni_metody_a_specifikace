\documentclass{article}
\usepackage[utf8]{inputenc}
\usepackage{float}
\usepackage{makecell}

\title{MI-FME Cvičení 3}
\author{Tomáš Chvosta}
\date{Únor 2020}

\setcounter{secnumdepth}{-2} % no numbered sections
\begin{document}

\maketitle

\section{Cvičení 3a}

\subsection{Zadání:}

Dokažte následující formuli:
$$[\neg[[r \lor s] \Rightarrow q ] \wedge [[r \lor s] \Rightarrow q]] \Rightarrow [[p \Rightarrow q] \wedge \neg [p \Rightarrow q]]$$

\subsection{Důkaz:}

Jelikož se jedná o implikaci, předpokládáme, že platí levá strana pravidla tedy konjunkce $\neg[[r \lor s] \Rightarrow q ]$ a $[[r \lor s] \Rightarrow q]]$. Tyto dva předpoklady představují $\bot$ a jelikož z $\bot$ plyne cokoliv, nezáleží na tom, co máme na pravé straně implikace a formule vždy platí.

\begin{table}[H]\centering

    \caption{Důkazová tabulka}

\begin{tabular}{|c|c|c|c}
    
    % \\ je novej radek | je cara \hline je horizontální čara
    
        \hline \textbf{Krok} & \textbf{Předpokládáme} & \textbf{Dokazujeme} \\ \hline \hline
    	1. & \makecell{$\neg[[r \lor s] \Rightarrow q ]$ \\ $[[r \lor s] \Rightarrow q]]$ \dots $\bot$} & \makecell{$\ \ [p \Rightarrow q]$ \\ $\neg [p \Rightarrow q]$} \\ \hline
    	
    
            
    	\end{tabular}
\end{table}

\section{Cvičení 3b}

\subsection{Zadání:}
Dokažte následující formuli:
$$[\neg p \Rightarrow p] \Rightarrow p$$

\subsection{Důkaz:}
Jelikož se jedná o implikaci, je předpoklad $[\neg p \Rightarrow p]$ a pokusíme se dokázat $p$. Použijeme Ratschanovo důkazní pravidlo, které říká, že když chceme dokázat $p$, pak můžeme nahradit $p$ za $\neg \neg p$ a následně použít pravidlo pro dokazování negací. Do seznamu předpokladů tedy přidáme $\neg p$ a pokusíme se najít spor. Předpoklad $[\neg p \Rightarrow p]$ říká, že musí platit $p$ jelikož máme v předpokladech $\neg p$, což je spor. Spor podle Ratschana dokončí úspěšně jakýkoliv důkaz.

\begin{table}[H]\centering

    \caption{Důkazová tabulka}

\begin{tabular}{|c|c|c|}
    
    % \\ je novej radek | je cara \hline je horizontální čara
    
        \hline \textbf{Krok} & \textbf{Předpokládáme} & \textbf{Dokazujeme} \\ \hline \hline
    	1. & $[\neg p \Rightarrow p]$ & $p$ \\ \hline
    	2. & $\neg p$ & $\neg \neg p$ tedy $p$ \\ \hline
    	3. & $p$ \dots $\bot$ & \\ \hline
    
            
    	\end{tabular}
\end{table}

\section{Cvičení 3c}

\subsection{Zadání:}

Dokažte následující formuli:
$$\neg [p \Rightarrow q] \Rightarrow [q \Rightarrow p]$$

\subsection{Důkaz:}

Jelikož se jedná o implikaci, je $\neg [p \Rightarrow q]$ předpoklad. Pokusíme se tedy dokázat $[q \Rightarrow p]$. Použijeme stejný postup a předpokládáme, že platí $q$. Nyní by se mohlo hodit dokázat, že platí $[p \Rightarrow q]$. Jako lemma tedy zvolíme $[p \Rightarrow q]$ a díky předpokladu $q$ je jasné, že toto lemma platí. Můžeme tedy přidat předpoklad $[p \Rightarrow q]$, což společně s předpokladem $\neg [p \Rightarrow q]$ vytvoří $\bot$, ze které plyne cokoliv.

\begin{table}[H]\centering

    \caption{Důkazová tabulka}

\begin{tabular}{|c|c|c|}
    
    % \\ je novej radek | je cara \hline je horizontální čara
    
        \hline \textbf{Krok} & \textbf{Předpokládáme} & \textbf{Dokazujeme} \\ \hline \hline
    	1. & $\neg [p \Rightarrow q]$ & $[q \Rightarrow p]$ \\ \hline
    	2. & $q$ & $p$ \\ \hline
    	3. &  & lemma $[p \Rightarrow q]$ \\ \hline
    	4. & $[p \Rightarrow q]$ \dots $\bot$  &  \\ \hline
    
            
    	\end{tabular}
\end{table}

\section{Cvičení 3d}

\subsection{Zadání:}

Dokažte následující formuli:
$$[p \Rightarrow [[q \lor r] \wedge \neg q \wedge \neg r]] \Rightarrow \neg p$$

\subsection{Důkaz:}

Nejprve předpokládejme $[p \Rightarrow [[q \lor r] \wedge \neg q \wedge \neg r]]$ a dokažme $\neg p$. To uděláme tak, že předpokládáme $p$ a zkusíme najít spor. Pokud platí $p$, tak můžeme usoudit $[[q \lor r] \wedge \neg q \wedge \neg r]]$. Z předchozího předpokladu můžeme usoudit, že platí $\neg q$, $\neg r$ a také $[q \lor r]$. Pokud víme, že platí disjunkce $[q \lor r]$, pak postupujeme tak, že nejprve předpokládáme $q$ a následně dokončíme důkaz a poté předpokládáme $r$ a následně dokončíme důkaz. V obou dvou případech získáme $\bot$, čímž získáváme spor a důkaz je úspěšně dokončen.

\begin{table}[H]\centering

    \caption{Důkazová tabulka}

\begin{tabular}{|c|c|c|}
    
    % \\ je novej radek | je cara \hline je horizontální čara
    
        \hline \textbf{Krok} & \textbf{Předpokládáme} & \textbf{Dokazujeme} \\ \hline \hline
    	1. & $[p \Rightarrow [[q \lor r] \wedge \neg q \wedge \neg r]]$ & $\neg p$ \\ \hline
    	2. & $p$ & hledáme spor  \\ \hline
    	3. & $[[q \lor r] \wedge \neg q \wedge \neg r]]$ & hledáme spor  \\ \hline
    	4. & \makecell{$\neg q$ \\ $\neg r$ \\ $[q \lor r]$} & hledáme spor  \\ \hline
    	5a. & $q$ \dots $\bot$ & hledáme spor  \\ \hline
    	5b. & $r$ \dots $\bot$ & hledáme spor  \\ \hline
    
            
    	\end{tabular}
\end{table}

\section{Cvičení 3e}

\subsection{Zadání:}

Dokažte následující formuli:
$$q \Rightarrow [[p \wedge q] \lor [\neg p \wedge q]]$$

\subsection{Důkaz:}

Na počátku předpokládáme, že platí $q$ a dokážeme $[[p \wedge q] \lor [\neg p \wedge q]]$. To můžeme udělat například tak, že předpokládáme $\neg [p \wedge q]$ a dokážeme $[\neg p \wedge q]$ (nebo klidně obráceně, avšak toto je výhodnější a snazší). Při důkazu konjunkce je podtřeba dokázat $\neg p$ i $q$. Při dokazování $q$ využijeme toho, že máme $q$ v předpokladech a tedy je triviálně dokázáno. Při dokazování $\neg p$ předpokládáme, že platí $p$ a najdeme spor. Ten jsme však již vytvořili přidáním $p$ do předpokladů, poněvadž jistě platí $[p \wedge q]$ díky předpokladům $p$ a $q$ a tedy spolu s $\neg [p \wedge q]$ získáme $\bot$, čímž je důkaz úspěšně dokončen.

\begin{table}[H]\centering

    \caption{Důkazová tabulka}

\begin{tabular}{|c|c|c|}
    
    % \\ je novej radek | je cara \hline je horizontální čara
    
        \hline \textbf{Krok} & \textbf{Předpokládáme} & \textbf{Dokazujeme} \\ \hline \hline
    	1. & $q$ & $[[p \wedge q] \lor [\neg p \wedge q]]$ \\ \hline
    	2. & $\neg [p \wedge q]$  & $[\neg p \wedge q]$  \\ \hline
    	3. &  & $q$  \\ \hline
    	4. &  & \makecell{$\neg p$ \\ $q$}  \\ \hline
    	5. & $p$  & hledáme spor  \\ \hline
    	6. & & lemma $[p \wedge q]$ \\ \hline
    	7. & $[p \wedge q]$ \dots $\bot$  & hledáme spor  \\ \hline
    	
    
            
    	\end{tabular}
\end{table}

\section{Cvičení 3f}

\subsection{Zadání:}

Dokažte následující formuli:
$$\neg [p \wedge q] \Rightarrow [\neg p \lor \neg q]$$

\subsection{Důkaz:}

Nejprve předpokládáme, že platí $\neg [p \wedge q]$ a dokážeme $[\neg p \lor \neg q]$. To dokážeme tak, že předpokládáme, že platí $p$ a dokážeme $\neg q$. To dokážeme tak, že předpokládáme $q$ a najdeme spor. Nyní díky předpokladům $p$ a $q$ můžeme usoudit, že platí lemma $[p \wedge q]$, což však společně s $\neg [p \wedge q]$ tvoří $\bot$. Podobně bychom mohli zvolit v druhém kroku $q$ místo $p$ a dostali bychom stejný výsledek.

\begin{table}[H]\centering

    \caption{Důkazová tabulka}

\begin{tabular}{|c|c|c|}
    
    % \\ je novej radek | je cara \hline je horizontální čara
    
        \hline \textbf{Krok} & \textbf{Předpokládáme} & \textbf{Dokazujeme} \\ \hline \hline
    	1. & $\neg [p \wedge q]$ & $[\neg p \lor \neg q]$  \\ \hline
    	2a. & $p$ & $\neg q$  \\ \hline
    	2b. & $q$ & $\neg p$  \\ \hline
    	3a. & $q$ & hledáme spor  \\ \hline
    	3b. & $p$ & hledáme spor  \\ \hline
    	4. &  & lemma $[p \wedge q]$  \\ \hline
    	5. & $[p \wedge q]$ \dots $\bot$ & hledáme spor  \\ \hline
            
    	\end{tabular}
\end{table}

\subsection{Poznámky:}
V tabulce buď zvolíme variantu a) nebo b), ale ne obojí!

\section{Cvičení 3g}

\subsection{Zadání:}

Dokažte následující formuli:
$$[[p \wedge q] \Rightarrow r] \Rightarrow [[p \Rightarrow r] \lor [q \Rightarrow r]]$$

\subsection{Důkaz:}

Nejprve předpokládejme, že platí $[[p \wedge q] \Rightarrow r]$ a dokážeme $[[p \Rightarrow r] \lor [q \Rightarrow r]]$. V tomto kroku máme opět dvě volby jako v předchozí úloze, nyní budeme předpokládat, že platí $\neg [q \Rightarrow r]$ a pokusíme se dokázat $p \Rightarrow r$. To dokážeme tak, že předpokládáme $p$ a dokazujeme $r$. Nyní se hodí využít faktu, že $r$ je ekvivalentní s $\neg \neg r$. Následně můžeme předpokládat $\neg r$ a poté se pokusit najít spor. Aby platil předpoklad $[p \wedge q] \Rightarrow r$ a zároveň předpoklad $\neg r$, musí platit $\neg [p \wedge q]$. Stejně tak, aby platil předpoklad $\neg [q \Rightarrow r]$ a zároveň $\neg r$, musí platit $q$. Nyní díky předpokladům $p$ a $q$ můžeme dokázat lemma $[p \wedge q]$, což nám vytvoří krásný spor s předpokladem $\neg [p \wedge q]$ a vznikne $\bot$, což úspěšně dokončí důkaz.

\begin{table}[H]\centering

    \caption{Důkazová tabulka}

\begin{tabular}{|c|c|c|}
    
    % \\ je novej radek | je cara \hline je horizontální čara
    
        \hline \textbf{Krok} & \textbf{Předpokládáme} & \textbf{Dokazujeme} \\ \hline \hline
    	1. & $[p \wedge q] \Rightarrow r$ & $[p \Rightarrow r] \lor [q \Rightarrow r]$ \\ \hline
    	2. & $\neg [q \Rightarrow r]$ & $p \Rightarrow r$ \\ \hline
    	3. & $p$ & $r$ \\ \hline
    	4. &  & $\neg \neg r$ \\ \hline
    	5. & $\neg r$ & hledáme spor \\ \hline
    	6. & $\neg [p \wedge q]$ & hledáme spor \\ \hline
    	7. & $q$ & hledáme spor \\ \hline
    	8. & $[p \wedge q] \dots \bot$ & hledáme spor \\ \hline
            
    	\end{tabular}
\end{table}

\section{Cvičení 3h}

\subsection{Zadání:}

Dokažte následující formuli:
$$ [p \wedge q] \Rightarrow \neg [\neg p \lor \neg q]$$

\subsection{Důkaz:}

Na začátku předpokládejme $p \wedge q$ a dokažme $\neg [\neg p \lor \neg q]$. Z předpokladu $p \wedge q$ můžeme usoudit $p$ a $q$. V dalším kroku předpokládejme, že platí $\neg p \lor \neg q$ a pokusíme se najít spor. Pokud víme, že platí $\neg p \lor \neg q$, můžeme nejprve předpokládat $\neg p$ a dokončit důkaz a poté předpokládat $\neg q$ a dokončit důkaz. V obou případech nalezneme spor a vznikne $\bot$, čímž je důkaz úspěšně dokončen.

\begin{table}[H]\centering

    \caption{Důkazová tabulka}

\begin{tabular}{|c|c|c|}
    
    % \\ je novej radek | je cara \hline je horizontální čara
    
        \hline \textbf{Krok} & \textbf{Předpokládáme} & \textbf{Dokazujeme} \\ \hline \hline
    	1. & $p \wedge q$ & $\neg [\neg p \lor \neg q]$ \\ \hline
    	2. & \makecell{$p$ \\ $q$} &  \\ \hline
    	3. & $\neg p \lor \neg q$ & hledáme spor \\ \hline
    	4. & $\neg p$ \dots $\bot$ & hledáme spor \\ \hline
    	5. & $\neg q$ \dots $\bot$ & hledáme spor \\ \hline
            
    	\end{tabular}
\end{table}

\section{Cvičení 3i}

\subsection{Zadání:}

Dokažte následující formuli:
$$[p \Rightarrow q] \lor [q \Rightarrow r]$$

\subsection{Důkaz:}

Jelikož máme dokázat disjunkci, můžeme postupovat tak, že předpokládáme, že levá část disjunkce neplatí a dokážeme pravou stranu. V tomto konkrétním případě to znamená předpokládat, že platí $\neg [p \Rightarrow q]$ a dokázat $q \Rightarrow r$. Pokračujeme předpokladem, že platí $q$ a máme dokázat $r$. Díky předpokladu $q$ si můžeme dokázat lemma $[p \Rightarrow q]$, díky platnosti $q$ bude lemma vždy platit bez ohledu na to, jestli platí $p$. Nyní se v předpokladech ocitla formule $[p \Rightarrow q]$, která nám vytvoří spor s prvním předpokladem. Z toho plyne, že první část počáteční formule $[p \Rightarrow q]$ musí platit, čímž jsme důkaz úspěšně dokončili.

\begin{table}[H]\centering

    \caption{Důkazová tabulka}

\begin{tabular}{|c|c|c|}
    
    % \\ je novej radek | je cara \hline je horizontální čara
    
        \hline \textbf{Krok} & \textbf{Předpokládáme} & \textbf{Dokazujeme} \\ \hline \hline
    	1. & $\neg [p \Rightarrow q]$  & $q \Rightarrow r$ \\ \hline
    	2. & $q$  & $r$ \\ \hline
    	3. &   & lemma $p \Rightarrow q$ \\ \hline
    	4. & $[p \Rightarrow q]$ \dots $\bot$  & $r$ \\ \hline
            
    	\end{tabular}
\end{table}

\end{document}
