\documentclass{article}
\usepackage[utf8]{inputenc}
\usepackage{float}

\title{MI-FME Cvičení 3}
\author{Tomáš Chvosta}
\date{Únor 2020}

\setcounter{secnumdepth}{-2} % no numbered sections
\begin{document}

\maketitle

\section{Cvičení 3a}

\subsection{Zadání:}

\subsection{Důkaz:}

\section{Cvičení 3b}

\subsection{Zadání:}
Dokažte následující formuli:
$$[\neg p \Rightarrow p] \Rightarrow p$$

\subsection{Důkaz:}
Jelikož se jedná o implikaci, je předpoklad $[\neg p \Rightarrow p]$ a pokusíme se dokázat $p$. Použijeme Ratschanovo důkazní pravidlo, které říká, že když chceme dokázat $p$, pak můžeme nahradit $p$ za $\neg \neg p$ a následně použít pravidlo pro dokazování negací. Do seznamu předpokladů tedy přidáme $\neg p$ a pokusíme se najít spor. Předpoklad $[\neg p \Rightarrow p]$ říká, že musí platit $p$ jelikož máme v předpokladech $\neg p$, což je spor. Spor podle Ratschana dokončí úspěšně jakýkoliv důkaz.

\begin{table}[H]\centering

    \caption{Důkazová tabulka}

\begin{tabular}{|c|c|c|}
    
    % \\ je novej radek | je cara \hline je horizontální čara
    
        \hline \textbf{Krok} & \textbf{Předpokládáme} & \textbf{Dokazujeme} \\ \hline \hline
    	1. & $[\neg p \Rightarrow p]$ & $p$ \\ \hline
    	2. & $\neg p$ & $\neg \neg p$ tedy $p$ \\ \hline
    	3. & $p$ \dots $\bot$ & \\ \hline
    
            
    	\end{tabular}
\end{table}

\section{Cvičení 3c}

\subsection{Zadání:}

\subsection{Důkaz:}

\section{Cvičení 3d}

\subsection{Zadání:}

\subsection{Důkaz:}

\section{Cvičení 3e}

\subsection{Zadání:}

\subsection{Důkaz:}

\section{Cvičení 3f}

\subsection{Zadání:}

\subsection{Důkaz:}

\section{Cvičení 3g}

\subsection{Zadání:}

\subsection{Důkaz:}

\section{Cvičení 3h}

\subsection{Zadání:}

\subsection{Důkaz:}

\section{Cvičení 3i}

\subsection{Zadání:}

\subsection{Důkaz:}

\end{document}
