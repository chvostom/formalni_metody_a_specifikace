\documentclass{article}
\usepackage[utf8]{inputenc}
\usepackage{float}
\usepackage{makecell}
\usepackage{hyperref}

\title{MI-FME Cvičení 0}
\author{Tomáš Chvosta}
\date{Únor 2020}

\setcounter{secnumdepth}{-2} % no numbered sections
\begin{document}

\maketitle

\section{Zadání}
Pokuste se zprovoznit software RISCAL, který naleznete na stránce, jejíž odkaz je zde: \href{https://www3.risc.jku.at/research/formal/software/RISCAL/}{\texttt{https://www3.risc.jku.at/research/formal/software/RISCAL/}}. Je zde několik způsobů, jak program zprovoznit (virtuální systém, instalace Java aplikace), můžete zvolit libovolnou z nich. Pokud během instalace narazíte na jakýkoliv problém, podívejte se do diskuzního fóra na Moodlu. Pokud je zde zmíněný váš problém, ale zatím není vyřešen, jednoduše k němu přidejte své jméno. Pokud zde problém není, přidejte popis vašeho problému.

Pokud se vám úspěšně podařilo nainstalovat program, potom se podívejte také do diskuzního fóra a pokuste se ostatním pomoci s jejich problémy. Ti, co pomohou ostatním, mohou být oceněni bonusovými body.

Upozornění: Tento úkol vypracujte co nejdříve! Pokud nezveřejníte svůj problém do 26. února, budu předpokládat, že jste program úspěšně nainstalovali.

\end{document}
