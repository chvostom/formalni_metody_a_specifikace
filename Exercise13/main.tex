\documentclass{article}
\usepackage[utf8]{inputenc}
\usepackage{float}
\usepackage{makecell}
\newcommand\tab[1][0.5cm]{\hspace*{#1}}
\newcommand{\N}{\mathbb{N}}
\usepackage{mathtools}
\usepackage{amssymb}

\usepackage{amsfonts}

\title{MI-FME Cvičení 13}
\author{Tomáš Chvosta}
\date{Duben 2020}

\setcounter{secnumdepth}{-2} % no numbered sections
\usepackage{czech}
\begin{document}

\maketitle

\section{Zadání}
Uvažujte proceduru s následujícím chováním:\newline\newline \textbf{procedure} $p(a, r, x)$\newline \textbf{Input:} $a \in \mathcal{N}, r \in \mathcal{N}, x \in \mathcal{N}$ s.t. $r \geq 0$ \newline \textbf{Output:} $a^*$ s.t. $a^* = a + rx$ \newline\newline Dokažte že je následující fragment programu zcela korektní (všechny proměnné jsou typu integer):\newline\newline \textbf{assume} $x \geq 10 \wedge k \geq 5 \newline p(x, 2, k) \newline @\ x \geq 15$ \newline\newline Dodržujte metody pro zpracování volání procedur, které jsou uvedené v přednáškách. Kromě toho také používejte dokazovací pravidla pro kvantifikátory. Můžete však libovolně využívat jakékoliv znalosti ohledně proměnných typů pole a integer.

\section{Řešení}

Nejprve vytvoříme logickou formuli pro proceduru $p$. Jelikož procedura mění $a$ a potřebujeme odlišný název pro vstupní a výstupní proměnnou, uvedeme vstupní a výstupní hodnotu zvlášť. Zárověň v logické formuli změníme význam $p$ na predikát:  
$$(\forall a, a^*,r, x \in \mathcal{N})((r \geq 0 \wedge p(a,a^*,r,x)) \Rightarrow a^* = a + rx)$$
Tuto formuli můžeme nyní brát jako předpoklad pro důkazy, ve kterých se bude vyskytovat naše procedura $p$. Pojďme si nyní převést do logické formule i náš program. Abychom něco takového mohli udělat, budeme nejprve potřebovat SSA formu:\newline\newline \textbf{assume} $x \geq 10 \wedge k \geq 5 \newline $\textbf{assume} $p(x,x_1 2, k) \newline @\ x_1 \geq 15$ \newline\newline
Je potřeba brát ohled na to, že $p$ v SSA formě představuje predikát. Logická formule z této SSA formy bude vypadat následovně:
$$(\forall x,x_1,k \in \mathcal{N})((x \geq 10 \wedge k \geq 5 \wedge p(x,x_1,2,k)) \Rightarrow x_1 \geq 15)$$
Z předpokladu, který popisuje proceduru $p$ pomocí logické formule po volbě $a \leftarrow x,\ a^* \leftarrow x_1, \ r \leftarrow 2,\ x \leftarrow k$, víme:
$$(\forall x,x_1,k \in \mathcal{N})((x \geq 10 \wedge k \geq 5 \wedge x_1 = x + 2k) \Rightarrow x_1 \geq 15)$$
Máme tedy tři předpolady $x \geq 10$, $k \geq 5$, $x_1 = x + 2k$ a máme dokázat $x_1 \geq 15$. Můžeme využít předpokladu $x_1 = x + 2k$ a upravit dokazovaný výraz na tvar $x + 2k \geq 15$. To můžeme snadno dokázat sporem. Předpokládáme tedy $\neg (x + 2k \geq 15)$, což je $x + 2k < 15$ a najdeme spor. Tento předpoklad můžeme upravit na tvar $2k < 15 - x$. Dále předpoklad $k \geq 5$ upravíme na tvar $2k \geq 10$. Složením předpokladů $2k < 15 - x$ a $2k \geq 10$ získáme $10 \leq 2k < 15 - x$ tedy $ 10 < 15 - x$, což můžeme upravit na tvar $x < 5$. To je však spor s předpokladem $x \geq 10$. Logická formule tedy platí a program je korektní.
\end{document}
