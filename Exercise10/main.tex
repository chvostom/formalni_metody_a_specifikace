\documentclass{article}
\usepackage[utf8]{inputenc}
\usepackage{float}
\usepackage{makecell}
\newcommand\tab[1][0.5cm]{\hspace*{#1}}

\title{MI-FME Cvičení 10}
\author{Tomáš Chvosta}
\date{Duben 2020}

\setcounter{secnumdepth}{-2} % no numbered sections
\usepackage{czech}
\begin{document}

\maketitle

Uvažujte následující program (všechny proměnné náleží množině přirozených čísel včetně nuly):
$\newline\newline x_0 \leftarrow x \newline i \leftarrow 0 \newline $ \textbf{while} $ i < n$ \textbf{do} $\newline \tab @\ x = x_0 + \sum_{k=0}^{i-1} a[k] \newline \tab x \leftarrow x + a[i] \newline \tab i \leftarrow i + 1 \newline @\ x = x_0 + \sum_{k=0}^{n-1} a[k]$

\section{Cvičení 10a}

\subsection{Zadání}
Napište všechny základní cesty programu uvedeného výše. Dbejte na to, aby cesty začínající v cyklu měly na začátku odpovídající předpoklad.

\subsection{Řešení}
Nejprve si můžeme všimnout, že program obsahuje while cyklus s podmínkou $i <n$, pomocí které můžeme náš program rozdělit na dvě základní cesty s~předpoklady, že tato podmínka platí a neplatí. Dále samotný cyklus obsahuje assertaci na svém začátku, která způsobí, že tělo cyklu také můžeme rozdělit na dvě základní cesty. Celkem tedy získáváme následující 4 základní cesty:\newline\newline \textbf{Základní cesta 1:} \newline\newline $x_0 \leftarrow x \newline i \leftarrow 0 \newline$ \textbf{assume} $\neg (i < n) \newline @\ x = x_0 + \sum_{k=0}^{n-1} a[k]$ \newline\newline Zde tedy předpokládáme, že podmínka cyklu není splněna a rovnou přejdeme k~assertaci na konci programu.\newline  \textbf{Základní cesta 2:} \newline\newline $x_0 \leftarrow x \newline i \leftarrow 0 \newline$ \textbf{assume} $(i < n) \newline @\ x = x_0 + \sum_{k=0}^{i-1} a[k]$ \newline\newline Zde naopak předpokládáme, že podmínka cyklu je splněna a přejdeme na začátek cyklu k assertaci, kterou tato základní cesta končí. \newline\newline \textbf{Základní cesta 3:} \newline\newline \textbf{assume} $x = x_0 + \sum_{k=0}^{i-1} a[k] \newline x \leftarrow x + a[i] \newline i \leftarrow i + 1 \newline$ \textbf{assume} $(i < n) \newline @\ x = x_0 + \sum_{k=0}^{i-1} a[k]$ \newline\newline Zde vidíme, že této části musela předcházet buď základní cesta 2, nebo 3 a~tedy musíme na začátek zapsat příslušný předpoklad. V tomto případě se jedná o~předpoklad $x = x_0 + \sum_{k=0}^{i-1} a[k]$. Na konci této základní cesty předpokládáme, že hodnota $i$ je pořád menší než hodnota $n$ a tedy přejdeme k assertaci na začátku cyklu, kterou tato základní cesta končí.  \newline\newline \textbf{Základní cesta 4:} \newline\newline \textbf{assume} $x = x_0 + \sum_{k=0}^{i-1} a[k] \newline x \leftarrow x + a[i] \newline i \leftarrow i + 1 \newline$ \textbf{assume} $\neg (i < n) \newline @\ x = x_0 + \sum_{k=0}^{n-1} a[k]$ \newline \newline Zde musí být na začátku opět stejný předpoklad jako u předchozí základní cesty. Na konci však předpokládáme, že podmínka cyklu není splněna, tedy hodnota $i$ není menší než hodnota $n$. Přejdeme tedy k assertaci na konci programu.

\section{Cvičení 10b}

\subsection{Zadání}
Pro každou základní cestu vytvořte ověřovací podmínku a případně se ji pokuste dokázat.

\subsection{Řešení}

Všechny základní cesty vytvořené v předchozí sekci byly převedeny do SSA formy a následně byla z této formy vytvořena logická formule. Zde už budu uvádět pouze logické formule.\newline\newline \textbf{Základní cesta 1:} 
$$(\forall x, x_0, i, n, a)([x_0 = x \wedge i = 0 \wedge \neg(i < n)] \Rightarrow x = x_0 + \sum_{k=0}^{n-1} a[k])$$ 
Máme tedy předpoklady $x_0 = x$, $i = 0$, $\neg(i < n)$ a pokusíme se dokázat $x = x_0 + \sum_{k=0}^{n-1} a[k]$. Předpoklad $\neg(i < n)$ můžeme zapsat jako $i \geq n$. Pomocí předpokladu $i = 0$ můžeme dosazením získat nový předpoklad $n \leq 0$, a~díky tomuto předpokladu víme, že horní hranice sumy $\sum_{k=0}^{n-1} a[k]$ bude menší než 0. To však nutně znamená, že $\sum_{k=0}^{n-1} a[k] = 0$ a že potřebujeme dokázat $x = x_0$, což je ale jeden z předpokladů. Ověřovací podmínka pro základní cestu 1 platí.
\newline\newline \textbf{Základní cesta 2:} $$(\forall x, x_0, i, n, a)([x_0 = x \wedge i = 0 \wedge i < n] \Rightarrow x = x_0 + \sum_{k=0}^{i-1} a[k])$$
Máme tedy předpoklady $x_0 = x$, $i = 0$, $i < n$ a pokusíme se dokázat pravou stranu formule $x = x_0 + \sum_{k=0}^{i-1} a[k]$. Můžeme využít předpokladu $i = 0$, který dosadíme do sumy a získáme $\sum_{k=0}^{-1} a[k] = 0$. Máme tedy dokázat $x = x_0$, což je ale jeden z předpokladů. Ověřovací podmínka pro základní cestu 2 platí.
\newline\newline \textbf{Základní cesta 3:} $$(\forall x, x_0, x_1, i, i_1, n, a)$$  $$([ x = x_0 + \sum_{k=0}^{i-1} a[k] \wedge x_1 = x + a[i] \wedge i_1 =  i + 1 \wedge i_1 < n]\Rightarrow x_1 = x_0 + \sum_{k=0}^{i_1 - 1} a[k])$$
Máme tedy předpoklady $x = x_0 + \sum_{k=0}^{i-1} a[k]$, $x_1 = x + a[i]$, $i_1 =  i + 1$, $i_1 < n$ a potřebujeme dokázat $x_1 = x_0 + \sum_{k=0}^{i_1 - 1} a[k]$. Můžeme využít předpokladů $x = x_0 + \sum_{k=0}^{i-1} a[k]$ a $x_1 = x + a[i]$, dosadit první do druhého, čímž získáme nový předpoklad $x_1 = x_0 + \sum_{k=0}^{i} a[k]$. Pokud použijeme předpoklad $i_1 = i + 1$ a dosadíme ho do dokazovaného výrazu $x_1 = x_0 + \sum_{k=0}^{i_1 - 1} a[k]$, získáme $x_1 = x_0 + \sum_{k=0}^{i} a[k]$, což už je ale jeden z předpokladů. Formule je tedy dokázána a ověřovací podmínka pro základní cestu 3 platí.
\newline\newline \textbf{Základní cesta 4:} \newpage $$(\forall x, x_0, x_1, i, i_1, n, a)$$  $$([ x = x_0 + \sum_{k=0}^{i-1} a[k] \wedge x_1 = x + a[i] \wedge i_1 =  i + 1 \wedge \neg (i_1 < n)]\Rightarrow x_1 = x_0 + \sum_{k=0}^{n - 1} a[k])$$ 
Máme tedy předpoklady $x = x_0 + \sum_{k=0}^{i-1} a[k]$, $x_1 = x + a[i]$, $i_1 =  i + 1$, $\neg (i_1 < n)$ a potřebujeme dokázat $x_1 = x_0 + \sum_{k=0}^{n - 1} a[k]$. Můžeme využít předpokladů $x = x_0 + \sum_{k=0}^{i-1} a[k]$ a $x_1 = x + a[i]$, dosadit první do druhého, čímž získáme nový předpoklad $x_1 = x_0 + \sum_{k=0}^{i} a[k]$. Dále můžeme podobným způsobem využít předpoklady $i_1 = i + 1$, $\neg(i_1 < n)$ a získat nový předpoklad $i \geq n - 1$. Nyní můžeme dosadit předpoklad $x_1 = x_0 + \sum_{k=0}^{i} a[k]$ do dokazované formule $x_1 = x_0 + \sum_{k=0}^{n - 1} a[k]$ a po úpravě získáme $\sum_{k=0}^{i} a[k] = \sum_{k=0}^{n - 1} a[k]$, což je třeba dokázat. To však platí pouze pro $i = n - 1$, ale dle předpokladu je $i \geq n - 1$. Pro $i > n - 1$ tedy formule nemusí platit a tedy ověřovací podmínka pro základní cestu 4 neplatí.

Můžeme si ukázat jednoduchý protipříklad, pro který formule neplatí. Například tento protipříklad:
$$\{x \mapsto 40, x_0 \mapsto 10, x_1 \mapsto 70, i \mapsto 2, i_1 \mapsto 3, n \mapsto 2, a \mapsto [10,20,30]\}$$ Toto ohodnocení proměnných splňuje levou stranu formule, avšak nesplňuje pravou stranu. Z toho můžeme usoudit, že formule neplatí.
\newline\newline 

\section{Cvičení 10c}

\subsection{Zadání}
Zkontrolujte všechny ověřovací podmínky. Pokud nějaká ověřovací podmínka neplatí, upravte assertaci v cyklu tak, aby všechny ověřovací podmínky platily. Neměňte však assertaci na konci programu. 

\subsection{Řešení}
V předchozí sekci si můžeme všimnout, že platí všechny podmínky kromě ověřovací podmínky u základní cesty 4. Tato ověřovací podmínka neplatí v případě, že $i > n - 1$ v průběhu cyklu. Podíváme-li se do původního programu, zjistíme, že k takové situaci nemůže dojít, jelikož je u while cyklu podmínka $i < n$. Stačí tedy tuto podmínku přidat i do assertace na začátku while cyklu. Program tedy po úpravě bude vypadat následovně:
$\newline\newline x_0 \leftarrow x \newline i \leftarrow 0 \newline $ \textbf{while} $ i < n$ \textbf{do} $\newline \tab @\ x = x_0 + \sum_{k=0}^{i-1} a[k] \wedge i < n \newline \tab x \leftarrow x + a[i] \newline \tab i \leftarrow i + 1 \newline @\ x = x_0 + \sum_{k=0}^{n-1} a[k]$. \newline\newline Pojďme se nyní podívat, jak se změní ověřovací podmínky v předchozí sekci po této úpravě.\newline\newline \textbf{Změna v základní cestě 1:} \newline Základní cesta 1 zůstává beze změny a tedy i její ověřovací podmínka. \newline\newline \textbf{Změna v základní cestě 2:} \newline V závěru základní cesty přibyde $\wedge (i < n)$, což je triviálně dokázáno díky předpokladu $ i < n$.  \newline\newline \textbf{Změna v základní cestě 3:} \newline Přibyde předpoklad $( i < n )$, který v našem případě nemá žádný vliv na platnost ověřovací podmínky.  \newline\newline \textbf{Změna v základní cestě 4:} \newline Přibyde předpoklad $( i < n )$, díky kterému ověřovací podmínka platí.

\end{document}
